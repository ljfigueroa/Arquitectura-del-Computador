\documentclass[10pt,oneside,a4paper]{article}
\usepackage{xltxtra} % Para usar características avanzadas de XeLaTeX, como elegir la fuente.
 
\usepackage{polyglossia} % Un sustituto de babel.
\setdefaultlanguage{spanish}
\usepackage{indentfirst} % Necesario para que sangre la primera línea de los primeros párrafos.
 
\defaultfontfeatures{Mapping=tex-text,Ligatures=Common} % Para que funcionen cosas como --- y las ligaduras.

\usepackage{hyperref}
%\usepackage{showframe}
\usepackage[top=2in, bottom=1.5in, left=1.5in, right=1in]{geometry}
%\usepackage{}
%\setmainfont{CMU Serif} % Ésta hay que descargarla. Si no, puedes usar la Linux Libertine.
%\setsansfont{CMU Sans Serif}
%\setmonofont{CMU Typewriter Text}
%\setmainfont{Linux Libertine 0}
%\setsansfont{Linux Biolinum O}
%\setmonofont{Inconsolata}


%\usepackage[pdftex]{graphicx}

\newcommand{\HRule}{\rule{\linewidth}{0.5mm}}

%\renewcommand\thesection{\arabic{section}}

\title{FDisk}
\author{Lauro Figueroa}
\date{1 de Enero de 1970}
\begin{document}
 
\begin{titlepage}
\begin{center}

% Upper part of the page. The '~' is needed because \\
% only works if a paragraph has started.
\includegraphics[width=0.15\textwidth]{./Logo.jpg}~\\[1cm]

\textsc{\LARGE Facultad de Ciencias Exactas, Ingeniería y Agrimensura}\\[1.5cm]

\textsc{\Large Trabajo final}\\[0.5cm]

% Title
\HRule \\[0.4cm]
{ \huge \bfseries Live FDisk \\[0.4cm] }

\HRule \\[1.5cm]

% Author and supervisor
\noindent
\begin{minipage}{0.4\textwidth}
\begin{flushleft} \large
\emph{Alumno:}\\
Lauro \textsc{Figueroa}
\end{flushleft}
\end{minipage}%
\begin{minipage}{0.4\textwidth}
\begin{flushright} \large
\emph{Profesores:} \\
Dr.Esteban \textsc{Ruiz}\\
Dr.Federico \textsc{Bergero}\\
Dr.Diego \textsc{Feroldi}
\end{flushright}
\end{minipage}

\vfill

% Bottom of the page
{\large \today}

\end{center}
\end{titlepage}

%\maketitle
\newpage 

\section*{Introdución}


\begin{center}
¿Por que elegí FDisk como trabajo final?
\end{center}
La elección fue sencilla, desde el comienzo de la carrera uno va aprendiendo
como un sistema operativo funciona, en cambio como es iniciado es algo
que desconocia completamente y de ahí se deriva el porque
de la elección de este trabajo final por sobre los otros.


%\setcounter{section}{0}
\section{Desarrollo y ejecución}

La ejecución de los objetivos los clasifique en tres partes bien definidas para
poder entender e implemtar de forma sencilla cada uno de ellos.

\subsection{Inicio y requerimientos para el arranque}
La primera parte consistió en entender como se inicializaba un procesador i386 y
como generar un programa assembler que sea ejecutable por el mismo. En un breve
resumen, el programa deberá ser un binario plano generado a partir de código assembler
16bits, con un tamaño de 512 bytes (tamaño de un sector) donde los dos ultimos bytes del mismo
deben ser el hexadecimal AA55h,
tambien llamado firma para que la BIOS lo reconosca como un gestor de arranque.
La BIOS carga el sector a la dirección de memoria 0x7C00 y luego salta a la
misma para ejecutar el programa.
\subsection{Impresión en pantalla.}
La primera vez que imprimí por pantalla fue un Hola mundo!
(como no podria ser de otra forma),
lo hice usando la llamada  INT 0x10 de la BIOS.
El algoritmo usado en escencialmente el mismo que el entregado ahora, la
funcion writeString leia de memoria la cadena asciz apuntada por \%si e iba
imprimiendo cada caracter hasta leer un 0. Pero
como debia que proveer varias funciones para imprimir en pantalla, cada
una con propositos diferentes, abstraje de la implementación de writeString
la función putchar, la cual imprime un caracter por pantalla. Esto a demas
de permitir implementar facilmente print-hex y print-num, hizo que agregar
caracteristica adicionales fuera sencillo, pues al transicionar a imprimir
por pantalla accediendo a la memoria de video solo tuve que cambiar la
función putchar.

Como mencione, una cadena de caracteres no es lo único que necesite
mostrar por pantalla, un problema que encontre
implementando la función que imprime números decimales de hasta 32bit de tamaño,
fue que el algoritmo por el cual obtengo los digitos del número usaba la
instrucción div del 8086 que toma sus valores en los registros
\%dx:\%ax como su dividendo de 32bits y guarda el modulo en \%dx y
el cociente en \%ax ambos de 16bits, imponiendo un limite sobre el tamaño del
dividendo para no generar un overflow al tratar de guardar el cociente
en el registro \%ax.
El algoritmo que implemente divide por diez el número a mostrar tantas veces
como cifras tenga, sea el siguiente un ejemplo del problema:
\%dx:\%ax posee FFFFFFFFh y el divisor es Ah (diez en decimal),
esto produce un error al dividir
pues el cociente es 19999999h y no entra en un registro de 16bits.

Para solucionarlo implemente la funcion div32 que se
encarga de dividir un numero 32bit pasado como argumento en los mismos registros
como lo hacia para div y como única restricción que el divisor debe estar en
el registro \%bx devolviendo el cociente como un número 32bits en \%dx:\%ax
y el resto en \%bx.

\subsection{Leer sectores del disco.}
La tercer parte fue la mas delicada, consistio en leer de distintos discos,
tanto el usb del cual se carga el gestor de arranque como del disco rigido
donde esta el MBR y lidiar con dos errores.
El primero fue generado por la instrucción INT 0x13 con el argumento
0 en el registro \%cx que es donde se especifica de que sector se
comienza a leer, levantando la bandera de acarreo despues de la operación
indicaba que un error se habia producido y 
analizando el valor devuelto en \%al y comparandolo con la lista de posibles
\href{http://www.pelletiernet.com/helppc/int_13-1.html}{errores},
arrojaba el error 'bad comand pass to the driver'
dado que se comienza del sector 1 usando la lectura en modo CHS.
Aunque el error mas dificil de diagnosticar fue el cometido por mi propia mano
al no calcular la direccion efectiva de memoria donde guardaba los datos que
estaba leyendo, los mismo se superponian con una parte del stack.
Para poder encontrarlo use las herramientas que ofrece gdb con qemu
en donde analizando la memoria donde se guardaba, note mi error.

\section{Tabla de particiones y su lugar en el esquema de desarrollo presentado.}

Imprimir la tabla de particiones fue sencillo una vez que tenia implementado
lo descripto anteriormente.
El diseño de la tabla impresa en pantalla fue directamente influenciado por
el como programa fdisk muestra en una terminal la misma informacion a
analizar.

Ahora mostrare como se imprime una entrada cualquiera
de las cuatro posibles de la tabla de particiones, ademas de explicitar si
cambie o hice abuso de la notación convencional.

El campo de columna Boot en el formato propuesto se obtione del primer
byte de la entrada en la tabla de
particiones, éste contiene la informacion del estado de la partición:

\begin{itemize}
\item 0x80 esta activa / es booteable, 
\item 0x00 esta inactiva / no es booteable
\item 0x01- 0x7F es una entrada invalida.
\end{itemize}

En la representacion que ofresco los muestro como 'B', 'I' y '-'
respectivamente.\\

La lectura del primer sector en la particion tanto como la cantidad de sectores,
lo hago usando el campo LBA y no CHS debido al limite de direccionamiento de 
CHS que es de 8GB por disco y en mi caso es de 1TB lo que hace imposible 
que se pueda guardar
esa información en los campos CHS y deba leer los campo LBA cuyo tamaño es
de 32bits, ie dos bytes, permitiendo representar hasta 2TB.

\section*{Bibliografia}


\href{http://susam.in/articles/boot-sector-code/}{Susam In}\\
\href{http://www.pelletiernet.com/helppc/int_13-2.html}{INT 13}\\
\href{http://en.wikipedia.org/wiki/Master_boot_record}{MBR wiki}

\end{document}
